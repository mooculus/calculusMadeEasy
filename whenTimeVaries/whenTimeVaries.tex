\documentclass{ximera}

\graphicspath{
  {./}
  {onDifferentDegreesOfSmallness/}
  {onRelativeGrowings/}
  {nextStageWhatToDoWithConstants/}
  {curvatureOfCurves/}               
}
\renewcommand{\d}{\mathop{}\!d}


\title{When Time Varies}

\begin{document}
\begin{abstract}
\end{abstract}
\maketitle

Some of the most important problems of the calculus
are those where time is the independent variable, and
we have to think about the values of some other
quantity that varies when the time varies. Some
things grow larger as time goes on; some other things
grow smaller. The distance that a train has got from
its starting place goes on ever increasing as time goes
on. Trees grow taller as the years go by. Which is
growing at the greater rate; a plant $12$ inches high
which in one month becomes $14$ inches high, or a
tree $12$ feet high which in a year becomes $14$ feet
high?

In this chapter we are going to make much use
of the word rate. Nothing to do with poor-rate, or
water-rate (except that even here the word suggests
a proportion�a ratio�so many pence in the pound).
Nothing to do even with birth-rate or death-rate,
though these words suggest so many births or deaths
per thousand of the population. When a motor-car
whizzes by us, we say: What a terrific rate! When
a spendthrift is flinging about his money, we remark
that that young man is living at a prodigious rate.

What do we mean by rate? In both these cases we
are making a mental comparison of something that is
happening, and the length of time that it takes to
happen. If the motor-car flies past us going $10$ yards
per second, a simple bit of mental arithmetic will
show us that this is equivalent�while it lasts�to a
rate of $600$ yards per minute, or over $20$ miles per
hour.

Now in what sense is it true that a speed of
$10$ yards per second is the same as $600$ yards
per minute? Ten yards is not the same as $600$ yards,
nor is one second the same thing as one minute.
What we mean by saying that the rate is the same,
is this: that the proportion borne between distance
passed over and time taken to pass over it, is the
same in both cases.

Take another example. A man may have only
a few pounds in his possession, and yet be able to
spend money at the rate of millions a year�provided
he goes on spending money at that rate for a few
minutes only. Suppose you hand a shilling over
the counter to pay for some goods; and suppose the
operation lasts exactly one second. Then, during
that brief operation, you are parting with your money
at the rate of $1$ shilling per second, which is the
same rate as �$3$ per minute, or �$180$ per hour, or
�$4320$ per day, or �$1,576,800$ per year! If you have
�$10$ in your pocket, you can go on spending money
at the rate of a million a year for just $5\frac{1}{4}$ minutes.

It is said that Sandy had not been in London

above five minutes when �bang went saxpence.� If
he were to spend money at that rate all day long,
say for $12$ hours, he would be spending $6$ shillings

an hour, or �$3$. $12$s . per day, or �$21$. $12$s . a week,
not counting the Sabbath.

Now try to put some of these ideas into differential
notation.

Let $y$ in this case stand for money, and let $t$ stand
for time.

If you are spending money, and the amount you
spend in a short time $dt$ be called $dy$, the rate of
spending it will be $\dfrac{dy}{dt}$, or rather, should be written
with a minus sign, as $-\dfrac{dy}{dt}$, because $dy$ is a decrement,
not an increment. But money is not a good example
for the calculus, because it generally comes and goes
by jumps, not by a continuous flow�you may earn
�$200$ a year, but it does not keep running in all
day long in a thin stream; it comes in only weekly,
or monthly, or quarterly, in lumps: and your expenditure
also goes out in sudden payments.

A more apt illustration of the idea of a rate is
furnished by the speed of a moving body. From
London (Euston station) to Liverpool is $200$ miles.
If a train leaves London at $7$ o'clock, and reaches
Liverpool at $11$ o'clock, you know that, since it has
travelled $200$ miles in $4$ hours, its average rate must
have been $50$ miles per hour; because $\frac{200}{4} = \frac{50}{1}$. Here
you are really making a mental comparison between

the distance passed over and the time taken to pass
over it. You are dividing one by the other. If $y$ is
the whole distance, and $t$ the whole time, clearly the
average rate is $\dfrac{y}{t}$. Now the speed was not actually
constant all the way: at starting, and during the
slowing up at the end of the journey, the speed was
less. Probably at some part, when running downhill,
the speed was over $60$ miles an hour. If, during
any particular element of time $dt$, the corresponding
element of distance passed over was $dy$, then at that
part of the journey the speed was $\dfrac{dy}{dt}$. The rate at
which one quantity (in the present instance, distance)
is changing in relation to the other quantity (in this
case, time) is properly expressed, then, by stating the
differential coefficient of one with respect to the other.
A velocity, scientifically expressed, is the rate at which
a very small distance in any given direction is being
passed over; and may therefore be written
\[
v = \dfrac{dy}{dt}.
\]

But if the velocity $v$ is not uniform, then it must
be either increasing or else decreasing. The rate at
which a velocity is increasing is called the acceleration.
If a moving body is, at any particular instant, gaining
an additional velocity $dv$ in an element of time $dt$,
then the acceleration $a$ at that instant may be written
\[
a = \dfrac{dv}{dt};
\]

but $dv$ is itself $d\left( \dfrac{dy}{dt} \right)$. Hence we may put
\[
a = \frac{d\left( \dfrac{dy}{dt} \right)}{dt};
\]
and this is usually written $a = \dfrac{d^2y}{dt^2}$;
or the acceleration is the second differential coefficient
of the distance, with respect to time. Acceleration is
expressed as a change of velocity in unit time, for
instance, as being so many feet per second per second;
the notation used being $\text{feet} � \text{second}^2$.

When a railway train has just begun to move, its
velocity $v$ is small; but it is rapidly gaining speed�it
is being hurried up, or accelerated, by the effort of the
engine. So its $\dfrac{d^2y}{dt^2}$ is large. When it has got up its
top speed it is no longer being accelerated, so that
then $\dfrac{d^2y}{dt^2}$ has fallen to zero. But when it nears its
stopping place its speed begins to slow down; may,
indeed, slow down very quickly if the brakes are put
on, and during this period of deceleration or slackening
of pace, the value of $\dfrac{dv}{dt}$, that is, of $\dfrac{d^2y}{dt^2}$ will be negative.

To accelerate a mass $m$ requires the continuous
application of force. The force necessary to accelerate
a mass is proportional to the mass, and it is also
proportional to the acceleration which is being imparted.
Hence we may write for the force $f$, the
expression
\begin{align*}
f &= ma;\\

 \text{or}\;\;  f &= m \frac{dv}{dt}; \\
 \text{or}\;\;  f &= m \frac{d^2y}{dt^2}.
\end{align*}

The product of a mass by the speed at which it is
going is called its momentum, and is in symbols $mv$.
If we differentiate momentum with respect to time
we shall get $\dfrac{d(mv)}{dt}$ for the rate of change of momentum.
But, since $m$ is a constant quantity, this
may be written $m \dfrac{dv}{dt}$, which we see above is the same
as $f$. That is to say, force may be expressed either
as mass times acceleration, or as rate of change of
momentum.

Again, if a force is employed to move something
(against an equal and opposite counter-force), it does
work; and the amount of work done is measured by
the product of the force into the distance (in its
own direction) through which its point of application
moves forward. So if a force $f$ moves forward
through a length $y$, the work done (which we may
call $w$) will be
\[
w = f � y;
\]
where we take $f$ as a constant force. If the force
varies at different parts of the range $y$, then we must
find an expression for its value from point to point.
If $f$ be the force along the small element of length $dy$,
the amount of work done will be $f � dy$. But as
$dy$ is only an element of length, only an element of
work will be done. If we write $w$ for work, then an
element of work will be $dw$; and we have
\begin{align*}
dw &= f � dy; \\
\end{align*}

which may be written
\begin{align*}
dw &= ma�dy; \\
\text{ or}\;
dw &= m \frac{d^2y}{dt^2}� dy; \\
 \text{or}\;
dw &= m \frac{dv}{dt}� dy.     \\
\end{align*}
Further, we may transpose the expression and write
\begin{align*}
\frac{dw}{dy} &= f.
\end{align*}

This gives us yet a third definition of force; that
if it is being used to produce a displacement in any
direction, the force (in that direction) is equal to the
rate at which work is being done per unit of length
in that direction. In this last sentence the word
rate is clearly not used in its time-sense, but in its
meaning as ratio or proportion.

Sir Isaac Newton, who was (along with Leibniz)
an inventor of the methods of the calculus, regarded
all quantities that were varying as flowing; and the
ratio which we nowadays call the differential coefficient
he regarded as the rate of flowing, or the
fluxion of the quantity in question. He did not use
the notation of the $dy$ and $dx$, and $dt$ (this was due
to Leibnitz), but had instead a notation of his own.
If $y$ was a quantity that varied, or �flowed,� then his
symbol for its rate of variation (or �fluxion�) was

$\dot{y}$. If $x$ was the variable, then its fluxion was called $\dot{x}$.
The dot over the letter indicated that it had been
differentiated. But this notation does not tell us
what is the independent variable with respect to
which the differentiation has been effected. When
we see $\dfrac{dy}{dt}$ we know that $y$ is to be differentiated with
respect to $t$. If we see $\dfrac{dy}{dx}$ we know that $y$ is to be
differentiated with respect to $x$. But if we see merely $\dot{y}$,
we cannot tell without looking at the context
whether this is to mean $\dfrac{dy}{dx}$ or $\dfrac{dy}{dt}$ or $\dfrac{dy}{dz}$, or what is
the other variable. So, therefore, this fluxional notation
is less informing than the differential notation,
and has in consequence largely dropped out of use.
But its simplicity gives it an advantage if only we
will agree to use it for those cases exclusively where
time is the independent variable. In that case $\dot{y}$ will
mean $\dfrac{dy}{dt}$ and $\dot{u}$ will mean $\dfrac{du}{dt}$;
and $\ddot{x}$ will mean $\dfrac{d^2x}{dt^2}$.

Adopting this fluxional notation we may write the
mechanical equations considered in the paragraphs
above, as follows:

\begin{tabular}{ l r }
distance & $x$ \\
velocity & $v = \dot{x}$ \\
acceleration & $a = \dot{v} = \ddot{x}$ \\
force & $f = m\dot{v} = m\ddot{x}$ \\
work & $w = x � m \ddot{x}$ \\
\end{tabular}





Examples
(1) A body moves so that the distance $x$ (in feet),
which it travels from a certain point $O$, is given by
the relation $x = 0.2t^2 + 10.4$, where $t$ is the time in
seconds elapsed since a certain instant. Find the
velocity and acceleration $5$ seconds after the body
began to move, and also find the corresponding values
when the distance covered is $100$ feet. Find also
the average velocity during the first $10$ seconds of
its motion. (Suppose distances and motion to the
right to be positive.)

   Now
   \[
x = 0.2t^2 + 10.4 \\
v = \dot{x}  = \frac{dx}{dt} = 0.4t;\quad\text{and}\quad
a = \ddot{x} = \frac{d^2x}{dt^2} = 0.4 = \text{constant.}
\]

When $t = 0$, $x = 10.4$ and $v = 0$. The body started
from a point $10.4$ feet to the right of the point $O$;
and the time was reckoned from the instant the
body started.

When $t = 5$, $v = 0.4 � 5 = 2 \text{ft./sec.}$; $a = 0.4 \text{ft./sec}^2$.


When $x = 100$, $100 = 0.2t^2 + 10.4$, or $t^2 = 448$,
and $t = 21.17 \text{sec.}$; $v = 0.4 � 21.17 = 8.468 \text{ft./sec.}$

When $t = 10$,
\begin{gather*}
\text{distance travelled} = 0.2 � 10^2 + 10.4 - 10.4 = 20 \text{ft.} \\
\text{Average velocity} = \tfrac{20}{10} = 2 \text{ft./sec.}
\end{gather*}

(It is the same velocity as the velocity at the middle
of the interval, $t = 5$; for, the acceleration being constant,
the velocity has varied uniformly from zero
when $t = 0$ to $4 \text{ft./sec.}$ when $t = 10$.)


(2) In the above problem let us suppose
\begin{gather*}
x = 0.2t^2 + 3t + 10.4.\\
v = \dot{x} = \dfrac{dx}{dt} = 0.4t + 3;\quad a = \ddot{x} = \frac{d^2x}{dt^2} = 0.4 = \text{constant}.
\end{gather*}

When $t = 0$, $x = 10.4$ and $v = 3$ ft./sec, the time is
reckoned from the instant at which the body passed a
point $10.4$ ft. from the point $O$, its velocity being then
already $3$ ft./sec. To find the time elapsed since it began
moving, let $v = 0$; then $0.4t + 3 = 0$, $t= -\frac{3}{.4}
= -7.5$ sec.
The body began moving $7.5$ sec. before time was
begun to be observed; $5$ seconds after this gives
$t = -2.5$ and $v = 0.4 � -2.5 + 3 = 2$ ft./sec.

When $x = 100$ ft.,
\[
100 = 0.2t^2 + 3t + 10.4;\quad \text{or } t^2 + 15t - 448 = 0;
\]
hence $t = 14.95$ sec., $v = 0.4 � 14.95 + 3 = 8.98$ ft./sec.

To find the distance travelled during the $10$ first
seconds of the motion one must know how far the
body was from the point $O$ when it started.

When $t = -7.5$,
\[
x = 0.2 � (-7.5)^2 - 3 � 7.5 + 10.4 = -0.85 \text{ft}.,
\]
that is $0.85$ ft. to the left of the point $O$.

Now, when $t = 2.5$,
\[
x = 0.2 � 2.5^2 + 3 � 2.5 + 10.4 = 19.15.
\]

So, in $10$ seconds, the distance travelled was
$19.15 + 0.85 = 20$ ft., and
\[
\text{the average velocity } = \tfrac{20}{10} = 2 \text{ ft./sec}.
\]

(3) Consider a similar problem when the distance
is given by $x = 0.2t^2 - 3t + 10.4$. Then $v = 0.4t - 3$,
$a = 0.4 = \text{constant}$. When $t = 0$, $x = 10.4$ as before, and

$v = -3$; so that the body was moving in the direction
opposite to its motion in the previous cases. As the
acceleration is positive, however, we see that this
velocity will decrease as time goes on, until it becomes
zero, when $v = 0$ or $0.4t - 3 = 0$; or $t = 7.5$ sec. After
this, the velocity becomes positive; and $5$ seconds
after the body started, $t = 12.5$, and
\[
v = 0.4 � 12.5 - 3 = 2 \text{ ft./sec}.
\]

When $x = 100$,
\[
100 = 0.2t^2 - 3t + 10.4,\quad \text{or } t^2 - 15t - 448 = 0, \\
 \text{and}\;
t = 29.95;\ v = 0.4 � 29.95 - 3 = 8.98 \text{ft./sec.}
\]

When $v$ is zero, $x = 0.2 � 7.5^2 - 3 � 7.5 + 10.4 = -0.85$,
informing us that the body moves back to $0.85$ ft.
beyond the point $O$ before it stops. Ten seconds later
\[
t = 17.5 \text{ and } x = 0.2 � 17.5^2 - 3 � 17.5 + 10.4 = 19.15.
\]
$\text{The distance travelled} = .85 + 19.15 = 20.0$, and the
average velocity is again $2$ ft./sec.

(4) Consider yet another problem of the same sort
with $x = 0.2t^3 - 3t^2 + 10.4$; $v = 0.6t^2 - 6t$; $a = 1.2t - 6$.
The acceleration is no more constant.

When $t = 0$, $x = 10.4$, $v = 0$, $a = -6$. The body is
at rest, but just ready to move with a negative
acceleration, that is to gain a velocity towards the
point $O$.

(5) If we have $x = 0.2t^3 - 3t + 10.4$, then $v = 0.6t^2 - 3$,
and $a = 1.2t$.

When $t = 0$, $x = 10.4$; $v = -3$; $a = 0$.

The body is moving towards the point $O$ with

a velocity of $3$ ft./sec., and just at that instant the
velocity is uniform.

We see that the conditions of the motion can always
be at once ascertained from the time-distance equation
and its first and second derived functions. In the
last two cases the mean velocity during the first
$10$ seconds and the velocity $5$ seconds after the start
will no more be the same, because the velocity is not
increasing uniformly, the acceleration being no longer
constant.

(6) The angle $\theta$ (in radians) turned through by a
wheel is given by $\theta = 3 + 2t - 0.1t^3$, where $t$ is the
time in seconds from a certain instant; find the
angular velocity $\omega$ and the angular acceleration $\alpha$,
(a ) after $1$ second; (b ) after it has performed one
revolution. At what time is it at rest, and how many
revolutions has it performed up to that instant?

Writing for the acceleration
\[
\omega =  \dot{\theta} = \dfrac{d\theta}{dt} = 2 - 0.3t^2,\quad
\alpha = \ddot{\theta} = \dfrac{d^2\theta}{dt^2} = -0.6t.
\]

When $t = 0$, $\theta = 3$; $\omega = 2$ rad./sec.; $\alpha = 0$.

When $t = 1$,
\[
\omega = 2 - 0.3 = 1.7 \text{rad./sec.};\quad \alpha = -0.6 \text{rad./sec}^2.
\]

This is a retardation; the wheel is slowing down.

After $1$ revolution
\[
\theta = 2\pi = 6.28;\quad 6.28 = 3 + 2t - 0.1t^3.
\]

By plotting the graph, $\theta = 3 + 2t - 0.1t^3$, we can get
the value or values of $t$ for which $\theta = 6.28$; these
are $2.11$ and $3.03$ (there is a third negative value).


When $t = 2.11$,
\begin{gather*}
\theta = 6.28;\quad\omega = 2 - 1.34 = 0.66 \text{ rad./sec.}; \\
\alpha = -1.27 \text{ rad./sec}^2. \\
\end{gather*}
When $t = 3.03$,
\begin{gather*}
\theta = 6.28;\quad \omega = 2 - 2.754 = -0.754 \text{ rad./sec.}; \\
\alpha = -1.82 \text{ rad./sec}^2.
\end{gather*}

The velocity is reversed. The wheel is evidently
at rest between these two instants; it is at rest when
$\omega = 0$, that is when $0 = 2 - 0.3t^3$, or when $t = 2.58 \text{sec.}$,
it has performed
\[
\frac{\theta}{2\pi}
  = \frac{3 + 2 � 2.58 - 0.1 � 2.58^3}{6.28} = 1.025 \text{ revolutions}.
\]


{\Large Exercises V\par}



\begin{problem}
If $y = a + bt^2 + ct^4$; find $\dfrac{dy}{dt} = \answer{2bt + 4ct^3}$ and $\dfrac{d^2y}{dt^2} = \answer{\dfrac{d^2y}{dt^2} = 2b + 12ct^2}$.
\end{problem}

\begin{problem}
A body falling freely in space describes in $t$ seconds
a space $s$, in feet, expressed by the equation
$s = 16t^2$. Draw a curve showing the relation between
$s$ and $t$. Also determine the velocity of the body at
the following times from its being let drop:
$t = 2$ seconds; Answer: $\answer{64}$
$t = 4.6$ seconds; Answer: $\answer{0.32}$
$t = 0.01$ second. Answer = $\answer{147.2}$ 
\end{problem}

\begin{problem}
 If $x = at - \frac{1}{2}gt^2$; find $\dot{x} = \answer{x = a - gt}$and $\ddot{x} = \answer{\ddot{x} = -g}$.
 \end{problem}

begin{problem}
If a body move according to the law
\[
s = 12 - 4.5t + 6.2t^2,
\]
find its velocity when $t = 4$ seconds; $s$ being in feet. \answer{$45.1$}
\end{problem}


\begin{problem}
Find the acceleration of the body mentioned in
the preceding example.
\answer{$12.4$}
\begin{question}
Is the acceleration the same for all values of $t$?
\begin{multipleChoice}
\choice[correct]{yes}
\choice{no}
end{multipleChoice}
end{question}
end{problem}


begin{problem}
The angle $\theta$ (in radians) turned through by
a revolving wheel is connected with the time $t$ (in
seconds) that has elapsed since starting; by the law
\[
\theta = 2.1 - 3.2t + 4.8t^2.
\]

\begin{problem}
Find the angular velocity (in radians per second) of
that wheel when $1\frac{1}{2}$ seconds have elapsed.
Angular velocity ${} = \answer{11.2}$ radians per second.
\begin{problem}
Find also its angular acceleration.
Angular acceleration ${}= \answer{9.6}$ radians per second per second
\end{problem}
\end{problem}


\begin{problem}
A slider moves so that, during the first part of
its motion, its distance $s$ in inches from its starting point is given by the expression
\[
s = 6.8t^3 - 10.8t;\quad\text{$t$ being in seconds}.
\]

Find the expression for the velocity at any time: $v = \answer{20.4t^2 - 10.8}$
\end{problem}


\begin{problem}
and the acceleration at any time: $a = \answer{40.8t}$
\begin{problem}
and hence find the velocity and the acceleration after $3$ seconds. Answer: /answer{$122.4} \text{in./sec}^2$
end{problem}
end{problem}
end{problem}

\begin{problem}
The motion of a rising balloon is such that its
height $h$, in miles, is given at any instant by the
expression $h = 0.5 + \frac{1}{10}\sqrt[3]{t-125}$; $t$ being in seconds.

Find an expression for the velocity and the acceleration at any time. Draw curves to show the variation of height, velocity and acceleration during the first ten minutes of the ascent.

Velocity: $v = \answer{\dfrac{1}{30 \sqrt[3]{(t - 125)^2}}}$
Acceleration: $a = \answer{- \dfrac{1}{45 \sqrt[3]{(t - 125)^5}}}$

\end{problem}

\begin{problem}
A stone is thrown downwards into water and its depth $p$ in metres at any instant $t$ seconds after
reaching the surface of the water is given by the
expression
\[
p = \frac{4}{4+t^2} + 0.8t - 1.
\]


Find an expression for the velocity and the acceleration at any time. 

Velocity: $v = \answer{0.8 - \dfrac{8t}{(4 + t^2)^2}}$
Acceleration: $v = \answer{0.8 - \dfrac{8t}{(4 + t^2)^2}}$
\begin{problem}
Find the velocity and acceleration after $10$ seconds.
Velocity: \answer{$0.7926$}
Acceleration: \answer{$0.00211$}
\end{problem}
\end{problem}

\begin{problem}
A body moves in such a way that the spaces
described in the time $t$ from starting is given by
$s = t^n$, where $n$ is a constant. Find the value of $n$ when the velocity is doubled from the $5$th to the $10$th second: \answer{$2$}
\begin{problem}
find it also when the velocity is numerically 
equal to the acceleration at the end of the $10$th second. \answer{$11$}
\end{problem}
\end{problem}

