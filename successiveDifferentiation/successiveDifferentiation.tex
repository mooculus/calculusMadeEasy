\documentclass{ximera}
\begin{document}
\title{Successive Differentiation}
\begin{abstract}
\end{abstract}
\maketitle

Let us try the effect of repeating several times over
the operation of differentiating a function (see \href{http://calculusmadeeasy.org/3.html#function}{here}).
Begin with a concrete case.

Let $y = x^5$.

\begin{align*}
&\text{First differentiation, }&&5x^4.&&\\
&\text{Second differentiation, }&&5 \times 4x^3&&= 20x^3. \\
&\text{Third differentiation, }&&5 \times 4 \times 3x^2&&= 60x^2. \\
&\text{Fourth differentiation, }&&5 \times 4 \times 3 � 2x&&= 120x.  \\
&\text{Fifth differentiation, }&&5 \times 4 \times 3 \times 2 \times 1&&= 120.   \\
&\text{Sixth differentiation, }&&                   &&= 0.
\end{align*}

There is a certain notation, with which we are
already acquainted (see \href{http://calculusmadeeasy.org/3.html#notation}{here}), used by some writers,
that is very convenient. This is to employ the
general symbol $f(x)$ for any function of $x$. Here
the symbol $f( )$ is read as ''function of,'' without
saying what particular function is meant. So the
statement $y=f(x)$ merely tells us that $y$ is a function
of $x$, it may be $x^2$ or $ax^n$, or $\cos x$ or any other complicated
function of $x$.

The corresponding symbol for the differential coefficient
is $f'(x)$, which is simpler to write than $\dfrac{dy}{dx}$.
This is called the ''derived function'' of $x$.


Suppose we differentiate over again, we shall get
the ''second derived function'' or second differential
coefficient, which is denoted by $f^{\prime\prime}(x)$; and so on.

Now let us generalize.

Let $y = f(x) = x^n$.
\begin{align*}
   \text{First differentiation,}\;      f^{\prime}(x) &= nx^{n-1}. \\
   \text{Second differentiation,}\;    f^{\prime\prime}(x) &= n(n-1)x^{n-2}. \\
   \text{Third differentiation,}\;    f^{\prime\prime\prime}(x) &= n(n-1)(n-2)x^{n-3}. \\
   \text{Fourth differentiation,}\;  f^{\prime\prime\prime\prime}(x) &= n(n-1)(n-2)(n-3)x^{n-4}. \\
    &\llap{\text{etc.,}} \text{ etc.}
\end{align*}

But this is not the only way of indicating successive
differentiations. For,
\begin{align*}
 \text{if the original function be}\;               y &= f(x);  \\
 \text{once differentiating gives}\;    \frac{dy}{dx} &= f'(x); \\
 \text{twice differentiating gives}\;   \frac{d\left(\dfrac{dy}{dx}\right)}{dx} &= f?(x);
\end{align*}
and this is more conveniently written as $\dfrac{d^2y}{(dx)^2}$, or
more usually $\dfrac{d^2y}{dx^2}$. Similarly, we may write as the
result of thrice differentiating, $\dfrac{d^3y}{dx^3} = f^{\prime\prime\prime}(x)$.


Now let us try $y = f(x) = 7x^4 + 3.5x^3 - \frac{1}{2}x^2 + x - 2$.
\begin{align*}
\frac{dy}{dx}     &= f^{\prime}(x) = 28x^3 + 10.5x^2 - x + 1, \\
\frac{d^2y}{dx^2} &= f^{\prime\prime}(x) = 84x^2 + 21x - 1,        \\
\frac{d^3y}{dx^3} &= f^{\prime\prime\prime}(x) = 168x + 21,             \\
\frac{d^4y}{dx^4} &= f^{\prime\prime\prime\prime}(x) = 168,                  \\
\frac{d^5y}{dx^5} &= f^{\prime\prime\prime\prime\prime}(x) = 0.
\end{align*}
In a similar manner if $y = \phi(x) = 3x(x^2 - 4)$,
\begin{align*}
\phi^{\prime}(x)    &= \frac{dy}{dx} = 3\bigl[x � 2x + (x^2 - 4) � 1\bigr] = 3(3x^2 - 4), \\
\phi^{\prime\prime}(x)   &= \frac{d^2y}{dx^2} = 3 � 6x = 18x, \\
\phi^{\prime\prime\prime}(x)  &= \frac{d^3y}{dx^3} = 18, \\
\phi^{\prime\prime\prime\prime}(x) &= \frac{d^4y}{dx^4} = 0.
\end{align*}

\textbf{Exercises IV}

Find $\dfrac{dy}{dx}$ and $\dfrac{d^2y}{dx^2}$ for the following expressions:

(1) $y = 17x + 12x^2$.

(2) $y = \dfrac{x^2 + a}{x + a}$. 

(3) $y = 1 + \dfrac{x}{1} + \dfrac{x^2}{1\times2} + \dfrac{x^3}{1\times2\times3} + \dfrac{x^4}{1\times2\times3\times4}$.

(4) Find the 2nd and 3rd derived functions in the Exercises III. (\href{http://calculusmadeeasy.org/6.html#examples2}{here}), No. 1 to No. 7, and in the
Examples given (\href{http://calculusmadeeasy.org/6.html#examples3}{here}), No. 1 to No. 7.

\textbf{Answers}

(1)$17 + 24x$; $24$.

(2)$\dfrac{x^2 + 2ax - a}{(x + a)^2}$;  $\dfrac{2a(a + 1)}{(x + a)^3}$.

(3)$1 + x + \dfrac{x^2}{1 � 2} + \dfrac{x^3}{1 \times 2 \times 3}$; $1 + x + \dfrac{x^2}{1\times 2}$.

{\emph{Exercises III}

(1) {\emph(a)} $\dfrac{d^2 y}{dx^2} = \dfrac{d^3 y}{dx^3} = 1 + x + \frac{1}{2}x^2 + \frac{1}{6} x^3 + \ldots$.
 
 {\emph(b)}  $2a$, $0$.
 
 {\emph(c)}  $2$, $0$.
  
 {\emph(d)}  $6x + 6a$, $6$.


(2) $-b$, $0$.

(3) $2$, $0$.

(4) $\begin{gathered}[t]
    56440x^3 - 196212x^2 - 4488x + 8192. \\
    169320x^2 - 392424x - 4488.
    \end{gathered}$
    
(5) $2$, $0$.

(6) $371.80453x$, $371.80453$.

(7) $\dfrac{30}{(3x + 2)^3}$, $-\dfrac{270}{(3x + 2)^4}$.


\emph{Examples}:

(1) $\dfrac{6a}{b^2} x$; $\dfrac{6a}{b^2}$.

(2) $\dfrac{3a \sqrt{b}} {2 \sqrt{x}} - \dfrac{6b \sqrt[3]{a}}{x^3}$,
$\dfrac{18b \sqrt[3]{a}}{x^4} - \dfrac{3a \sqrt{b}}{4 \sqrt{x^3}}$.

(3)
$\dfrac{2}{\sqrt[3]{\theta^8}} - \dfrac{1.056}{\sqrt[5]{\theta^{11}}}$,
$\dfrac{2.3232}{\sqrt[5]{\theta^{16}}} - \dfrac{16}{3 \sqrt[3]{\theta^{11}}}$.

(4)  $\begin{gathered}[t]
  810t^4 - 648t^3 + 479.52t^2 - 139.968t + 26.64. \\
  3240t^3 - 1944t^2 + 959.04t - 139.968.
  \end{gathered}$

(5)  $12x + 2$, $12$.

(6) $6x^2 - 9x$, $12x - 9$.

(7)\begin{align*}
&\dfrac{3}{4} \left(\dfrac{1}{\sqrt{\theta}} + \dfrac{1}{\sqrt{\theta^5}}\right)
+\dfrac{1}{4} \left(\dfrac{15}{\sqrt{\theta^7}} - \dfrac{1}{\sqrt{\theta^3}}\right). \\
&\dfrac{3}{8} \left(\dfrac{1}{\sqrt{\theta^5}} - \dfrac{1}{\sqrt{\theta^3}}\right)
-\dfrac{15}{8}\left(\dfrac{7}{\sqrt{\theta^9}} + \dfrac{1}{\sqrt{\theta^7}}\right).
\end{align*}


\end{document}
