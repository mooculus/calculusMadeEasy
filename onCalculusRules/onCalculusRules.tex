\documentclass{ximera}



\begin{document}

\chapter{Sums, Differences, Products and Quotients}

We have learned how to differentiate simple algebraical functions such as $x^2+c$
or $ax^4$, and we have now to consider how to tackle the sum of two or more functions.

For instance, let

$$
y=(x2+c)+(ax4+b);
$$

what will its $\dfrac{dy}{dx}$ be? How are we to go to work on this new job?

The answer to this question is quite simple: just differentiate them, one after the other, thus:

$$
\dfrac{dy}{dx}=2x+4ax^3.\quad \text{(Ans.)}
$$

If you have any doubt whether this is right, try a more general case, working it by first principles. And this is 
the way. Let $y=u+v$, where u is any function of $x$, and $v$ any other function of $x$. Then, letting $x$ increase 
to $x+dx$, $y$ will increase to $y+dy$; and u will increase to $u+du$; and $v$ to $v+dv$.

And we shall have:

$$
y+dy=u+du+v+dv.
$$

Subtracting the original $y=u+v$, we get
$$
dy=du+dv,
$$

and dividing through by $dx$, we get:

$$
\dfrac{dy}{dx}=\dfrac{du}{dx}+\dfrac{dv}{dx}.
$$

This justifies the procedure. You differentiate each function separately and add the results. So if now we take 
the example of the preceding paragraph, and put in the values of the two functions, we shall have, using the notation 
shown (\ref{section}),

$$
\dfrac{dy}{dx}=\dfrac{d(x^2+c)}{dx}=2x+\dfrac{d(ax^4+b)}{dx}+4ax^3,
$$

exactly as before.

If there were three functions of $x$, which we may call $u$, $v$ and $w$, so that

\begin{align*}
y &=u+v+w;\\
\text{then}\;\dfrac{dy}{dx} &=\dfrac{du}{dx}+\dfrac{dv}{dx}+\dfrac{dw}{dx}.
\end{align*}

As for subtraction, it follows at once; for if the function $v$
had itself had a negative sign, its differential coefficient would also be negative; so that by differentiating

\begin{align*}
y &=u-v,\\ 
\text{we should get}\; \dfrac{dy}{dx} &=\dfrac{du}{dx}−\dfrac{dv}{dx}.
\end{align*}

But when we come to do with \emph{Products}, the thing is not quite so simple.

Suppose we were asked to differentiate the expression

$$
y=(x^2+c)\times (ax^4+b),
$$

what are we to do? The result will certainly not be $2x\times 4ax^3$; for it is easy to see that neither 
$c \times ax^4$, nor $x^2\times b$, would have been taken into that product.

Now there are two ways in which we may go to work.

\emph{First way}. Do the multiplying first, and, having worked it out, then differentiate.

Accordingly, we multiply together $x^2+c$ and $ax^4+b$.

This gives $ax^6+acx^4+bx^2+bc$.


Now differentiate, and we get:

$$
\dfrac{dy}{dx}=6ax^5+4acx^3+2bx.
$$

Second way. Go back to first principles, and consider the equation

$$
y=u\times v;
$$

where $u$ is one function of $x$, and $v$ is any other function of $x$. Then, if $x$ grows to be $x+dx$; and $y$ 
to $y+dy$; and $u$ becomes $u+du$, and $v$ becomes $v+dv$, we shall have:

\begin{align*}
y+dy &=(u+du)\times (v+dv)\\
   &=u\cdot v+u\cdot dv+v\cdot du+du\cdot dv.
\end{align*}

Now $du\cdot dv$
is a small quantity of the second order of smallness, and therefore in the limit may be discarded, leaving

$$
y+dy=u\cdot v+u\cdot dv+v\cdot du.
$$

Then, subtracting the original $y=u\cdot v$, we have left

$$
dy=u\cdot dv+v\cdot du;
$$

and, dividing through by $dx$, we get the result:

$$
\dfrac{dy}{dx}=u\dfrac{dv}{dx}+v\dfrac{du}{dx}.
$$

This shows that our instructions will be as follows: \emph{To differentiate the product of two functions, 
multiply each function by the differential coefficient of the other, and add together the two products so obtained}.

You should note that this process amounts to the following: Treat $u$ as constant while you differentiate $v$; 
then treat $v$ as constant while you differentiate $u$; and the whole differential coefficient $\frac{dy}{dx}$

will be the sum of these two treatments.

Now, having found this rule, apply it to the concrete example which was considered above.

We want to differentiate the product

$$
(x^2+c)\times (ax^4+b).
$$

Call $(x^2+c)=u$; and $(ax^4+b)=v$.

Then, by the general rule just established, we may write:

\begin{align*}
\dfrac{dy}{dx} &=(x^2+c)\dfrac{d(ax^4+b)}{dx}+(ax^4+b)\dfrac{d(x^2+c)}{dx}\\
 &=(x^2+c)4ax^3=4ax^5+4acx^3=6ax^5+(ax^4+b)2x\\
\dfrac{dy}{dx} &=4ax^5+4acx^3+2ax^5+2bx,
\end{align*}

exactly as before.

Lastly, we have to differentiate \emph{quotients}.

Think of this example, $y=\frac{bx^5+c}{x^2+a}$. In such a case it is no use to try to work out the division 
beforehand, because $x^2+a$ will not divide into $bx^5+c$, neither have they any common factor. So there is 
nothing for it but to go back to first principles, and find a rule. So we will put $y=uv$;
where $u$ and $v$ are two different functions of the independent variable $x$. Then, when $x$ becomes $x+dx$, 
$y$ will become $y+dy$; and $u$ will become $u+du$; and $v$ will become $v+dv$. So then

$$
y+dy=\dfrac{u+du}{v+dv}.
$$

Now perform the algebraic division, thus:


image missing


As both these remainders are small quantities of the second order, they may be neglected, and the division may 
stop here, since any further remainders would be of still smaller magnitudes.

So we have got:

$$
y+dy =\dfrac{u}{v}+\dfrac{du}{v}−\dfrac{u\cdot dv}{v^2};
$$

which may be written

$$
=\dfrac{u}{v}+\dfrac{v\cdot du−u\cdot dv}{v^2}.
$$

Now subtract the original $y=\frac{u}{v}$, and we have left:

\begin{align*}
dy &=\dfrac{v\cdot du-u\cdot dv}{v^2}\\
\text{whence}\quad \dfrac{dy}{dx}=\dfrac{v\cdot du−u\cdot dv}{v^2}.
\end{align*}

This gives us our instructions as to \emph{how to differentiate a quotient of two functions. Multiply the divisor 
function by the differential coefficient of the dividend function; then multiply the dividend function by the 
differential coefficient of the divisor function; and subtract. Lastly divide by the square of the divisor function}.

Going back to our example $y=\frac{bx^5+c}{x^2+a}$,

\begin{align*}
\text{write}\quad bx^5+c &=u;\\
\text{and}\quad x^2+a &=v.
\end{align*}

Then

\begin{align*}
\dfrac{dy}{dx} &=\dfrac{(x^2+a)\dfrac{d(bx^5+c)}{dx}−(bx^5+c)\dfrac{d(x^2+a)}{dx}}{(x^2+a)^2}\\
 &=\dfrac{(x^2+a)(5bx^4)−(bx^5+c)(2x)}{(x^2+a)^2},\\
 &=\dfrac{3bx^6+5abx^4−2cx}{(x^2+a)^2}.\quad\text{(Answer.)}
\end{align*}

The working out of quotients is often tedious, but there is nothing difficult about it.

Some further examples fully worked out are given hereafter.

\par\noindent
(1) Differentiate  $y=\dfrac{a}{b^2}x^3−\dfrac{a^2}{b}x+\dfrac{a^2}{b^2}$.


Being a constant, $\frac{a^2}{b^2}$
vanishes, and we have

$$
\dfrac{dy}{dx}=\dfrac{a}{b^2}\times 3\times x^{3−1}−\dfrac{a^2}{b}\times 1\times x^{1−1}.
$$

But $x^{1−1}=x^0=1$; so we get:

$$
\dfrac{dy}{dx}=\dfrac{3a}{b^2}x^2−\dfrac{a^2}{b}.
$$

\par\noindent
(2) Differentiate $y=2a\sqrt{bx^3}−\dfrac{3b\sqrt[3]{a}}{x}−2\sqrt{ab}$.

Putting $x$ in the index form, we get

$$
y=2a\sqrt{bx^3}−\dfrac{3b\sqrt[3]{a}}{x}−2\sqrt{ab}.
$$

Now

\begin{align*}
\dfrac{dy}{dx} &=2a\sqrt{b}\times\dfrac{3}{2}\times x^{\frac{3}{2}−1}−3b\sqrt[3]{a}\times(−1)\times x^{−1−1};\\
\text{or},\quad \dfrac{dy}{dx} &=3a\sqrt{bx}+\dfrac{3b\sqrt[3]{a}}{x^2}.
\end{align*}

\par\noindent
(3) Differentiate $z=1.8\sqrt[3]{\dfrac{1}{\theta^2}}−\dfrac{4.4}{\sqrt[5]{\theta}}−27^{\circ}$.


This may be written: $z=1.8\theta^{−\frac{2}{3}}−4.4\theta^{−\frac{1}{5}}−27^{\circ}$.


The $27^{\circ}$ vanishes, and we have

\begin{align*}
\dfrac{dz}{d\theta} &=1.8\times−\dfrac{2}{3}\theta^{−\frac{2}{3}−1}−4.4\times\left(−\dfrac{1}{5}\right)\theta^{−\frac{1}{5}−1};\\
\text{or},\quad\dfrac{dz}{d\theta} &=−1.2\theta^{−\frac{5}{3}}+0.88\theta^{−\frac{6}{5}};\\
\text{or},\quad \dfrac{dz}{d\theta} &=\dfrac{0.88}{\sqrt[5]{\theta^6}}−\dfrac{1.2}{\sqrt[3]{\theta^5}}.
\end{align*}

\par\noindent
(4)  Differentiate $v=(3t^2−1.2t+1)^3$.


A direct way of doing this will be explained later (see here); but we can nevertheless manage it now without 
any difficulty.

Developing the cube, we get

$$
v=27t^6−32.4t^5+39.96t^4−23.328t^3−13.32t^2−3.6t+1;
$$

hence

$$
\dfrac{dv}{dt}=162t^5−162t^4+159.84t^3−69.984t^2+26.64t−3.6.
$$

\par\noindent
(5) Differentiate  $y=(2x−3)(x+1)^2$.

\begin{align*}
\dfrac{dy}{dx} &=(2x−3)\dfrac{d[(x+1)(x+1)]}{dx}+(x+1)^2\dfrac{d(2x-3)}{dx}\\
 &=(2x−3)\left[(x+1)\dfrac{d(x+1)}{dx}+(x+1)\dfrac{d(x+1)}{dx}\right] + (x+1)^2\dfrac{d(2x-3)}{dx}\\
 &=2(x+1)[(2x−3)+(x+1)]=2(x+1)(3x−2);
\end{align*}

or, more simply, multiply out and then differentiate.

\par\noindent
(6) Differentiate 

\begin{align*}
y &=0.5\left[x^3\dfrac{d(x−3)}{dx}+(x-3)\dfrac{d(x^3)}{dx}\right]\\
 &=0.5[x^3+(x−3)\times 3x^2]=2x^3−4.5x^2.
\end{align*}

Same remarks as for preceding example.

\par\noindent
(7) Differentiate $w=\left(\theta+\dfrac{1}{\theta}\right)\left(\sqrt{\theta}+\dfrac{1}{\sqrt{\theta}}\right)$.


This may be written

\begin{align*}
w &=(\theta+\theta^{−1})(\theta^{\frac{1}{2}}+\theta^{−\frac{1}{2}}).\\
\dfrac{dw}{d\theta} &=(\theta+\theta^{−1})\dfrac{d(\theta^{\frac{1}{2}}+\theta^{−\frac{1}{2}})}{d\theta}+(\theta^{\frac{1}{2}}+\theta^{−\frac{1}{2}})\dfrac{d(
\theta+\theta^{−1})}{d\theta}\\
&=(\theta+\theta^{−1})(\frac{1}{2}\theta^{−\frac{1}{2}}−\dfrac{1}{2}\theta^{−\frac{3}{2}})+(\theta^{\frac{1}{2}}+\theta^{−\frac{1}{2}})(1−\theta^{−2})\\
&=\dfrac{1}{2}(\theta^{\frac{1}{2}}+\theta^{−\frac{3}{2}}−\theta^{−\frac{1}{2}}−\theta^{−\frac{5}{2}})\\
&=\dfrac{3}{2}\left(\sqrt{\theta}−\dfrac{1}{\sqrt[5]{\theta}5}\right)+\dfrac{1}{2}\left(\dfrac{1}{\sqrt{\theta}}−\dfrac{1}{\sqrt{\theta^3}}\right).
\end{align*}

This, again, could be obtained more simply by multiplying the two factors first, and differentiating afterwards. 
This is not, however, always possible; see, for instance, here, example 8, in which the rule for differentiating 
a product \emph{must} be used.

\par\noindent
(8) Differentiate $y=\dfrac{a}{1+a\sqrt{x}+a^2x}$.

\begin{align*}
\dfrac{dy}{dx} &=\dfrac{(1+ax^{\frac{1}{2}}+a^2x)\times 0−a\frac{d(1+ax^{\frac{1}{2}}+a^2x)}{dx}}{(1+a\sqrt{x}+a\sqrt{a}+a^2x)^2}\\
 &=−\dfrac{a(\frac{1}{2}ax^{−\frac{1}{2}}+a^2)}{(1+ax^{\frac{1}{2}}+a^2x)^2}.\\
\end{align*}

\par\noindent
(9) Differentiate $y=\dfrac{x^2}{x^2+1}$.

$$
\dfrac{dy}{dx}=\dfrac{(x^2+1)\cdot 2x−x^2\times 2x}{(x^2+1)^2}=\dfrac{2x}{(x^2+1)^2}.
$$

\par\noindent
(10) Differentiate $y=\dfrac{ a+x^{\frac{1}{2}} }{ a−x^{\frac{1}{2}} }$.


In the indexed form, $y=\dfrac{ a+x^{\frac{1}{2}} }{ a−x^{\frac{1}{2}} }$.

\begin{align*}
\dfrac{dy}{dx} &=\dfrac{ (a−x^{\frac{1}{2}})(\frac{1}{2}x^{−\frac{1}{2}})−(a+x^{\frac{1}{2}})(−\frac{1}{2}x^{−\frac{1}{2}}) }{ (a−x^{\frac{1}{2}})^2 }\\
               &=\dfrac{ a−x^{\frac{1}{2}}+a+x^{\frac{1}{2}} }{ 2(a−x^{\frac{1}{2}})^2x^{\frac{1}{2}} };\\
\text{hence}\; \dfrac{dy}{dx} &=\dfrac{a}{ (a−\sqrt{x})^2\sqrt{x} }.
\end{align*}

\par\noindent
(11) Differentiate

\begin{align*}
\theta &=\dfrac{1-a\sqrt[3]{t^2}}{1+a\sqrt{t^3}}.\\
\text{Now}\quad \theta &=\dfrac{ 1−at^{\frac{2}{3}} }{ 1+at^{\frac{3}{2}} }.\\
\dfrac{d\theta}{dt} &=\dfrac{ (1+at^{\frac{3}{2}})(−\frac{2}{3}at^{−\frac{1}{3}})−(1−at^{\frac{2}{3}})\times\frac{3}{2}at^{\frac{1}{2}} }{ (1+at^{\frac{3}{2}})^2 }\\
                    &=\dfrac{5a^2\sqrt[6]{t^7}−\dfrac{4a}{\sqrt[3]{t}}−9a\sqrt{t}}{6(1+a\sqrt{t^3})^2}.
\end{align*}

\par\noindent
(12) A reservoir of square cross-section has sides sloping at an angle of $45^{\circ}$
with the vertical. The side of the bottom is $200$ feet. Find an expression for the quantity pouring in or out when 
the depth of water varies by 1 foot; hence find, in gallons, the quantity withdrawn hourly when the depth is reduced 
from $14$ to $10$ feet in $24$ hours.

The volume of a frustum of pyramid of height $H$, and of bases A and a, is $V=\dfrac{H}{3}(A+a+\sqrt{Aa})$. It is easily seen that, the slope being $45^{\circ}$, 
if the depth be $h$, the length of the side of the square surface of the water is $200+2h$ feet, so that the volume 
of water is

$$
\dfrac{h}{3}[200^2+(200+2h)^2+200(200+2h)]=40,000h+400h^2+\dfrac{4h^3}{3}.
$$

$$
\dfrac{dV}{dh}=40,000+800h+4h^2=\text{cubic feet per foot of depth variation.} 
$$

The mean level from $14$ to $10$ feet is $12$ feet, when $h=12$, $\dfrac{dV}{dh}=50,176$ cubic feet.

Gallons per hour corresponding to a change of depth of $4$ ft. in $24$ hours 

$$
=\dfrac{4\times 50,176\times 6.25}{24}=52,267\quad\text{gallons}.
$$

\par\noindent
(13) The absolute pressure, in atmospheres, $P$, of saturated steam at the temperature $t^{\circ}$ C. is given 
by Dulong as being $P=\left(\dfrac{40+t}{140}\right)^5$ as long as $t$ is above $80^{\circ}$. Find the 
rate of variation of the pressure with the temperature at $100^{\circ}$ C.

Expand the numerator by the binomial theorem %(see \ref{section}).

\begin{align*}
P &=\dfrac{1}{140^5}(40^5+5\times40^4t+10\times 40^3t^2+10\times 40^2t^3+5\times 40t^4+t^5);\\
\text{hence}\quad \dfrac{dP}{dt} &=\dfrac{1}{537,824\times10^5}(5\times40^4+20\times40^3t+30\times40^2t^2
+20\times40t^3+5t^4),
\end{align*}

when $t=100$ this becomes $0.036$

atmosphere per degree Centigrade change of temperature.



\end{document}
