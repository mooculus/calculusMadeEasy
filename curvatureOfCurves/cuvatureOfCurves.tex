Returning to the process of successive differentiation,
it may be asked: Why does anybody want to
differentiate twice over? We know that when the
variable quantities are space and time, by differentiating
twice over we get the acceleration of a
moving body, and that in the geometrical interpretation,
as applied to curves, $\dfrac{dy}{dx}$ means the <em>slope</em> of the
curve. But what can $\dfrac{d^2 y}{dx^2}$ mean in this case? Clearly
it means the rate (per unit of length $x$) at which the
slope is changing�in brief, it is <em>a measure of the
curvature of the slope</em>.

<a name="figure31">
<p><img src="33283-t/images/124a.pdf.png-1.png">
<a name="figure32">
<p><img src="33283-t/images/124b.pdf.png-1.png">

<p>Suppose a slope constant, as in <a href="#figure31">Figure 31</a>.

<p>Here, $\dfrac{dy}{dx}$ is of constant value.


<p>Suppose, however, a case in which, like <a href="#figure32">Figure 32</a>,
the slope itself is getting greater upwards, then
$\dfrac{d\left(\dfrac{dy}{dx}\right)}{dx}$, that is, $\dfrac{d^2y}{dx^2}$, will be <em>positive</em>.

<p>If the slope is becoming less as you go to the right
(as in <a href="10.html#figure14">Figure 14</a>), or as in <a href="#figure33">Figure 33</a>, then, even
though the curve may be going upward, since the
change is such as to diminish its slope, its $\dfrac{d^2y}{dx^2}$ will
be <em>negative</em>.

<a name="figure3">
<p><img src="33283-t/images/125a.pdf.png-1.png">

<p>It is now time to initiate you into another secret�how
to tell whether the result that you get by
�equating to zero� is a maximum or a minimum.
The trick is this: After you have differentiated
(so as to get the expression which you equate to
zero), you then differentiate a second time, and look
whether the result of the second differentiation is
<em>positive</em> or <em>negative</em>. If $\dfrac{d^2y}{dx^2}$ comes out <em>positive</em>, then
you know that the value of $y$ which you got was
a <em>minimum</em>; but if $\dfrac{d^2y}{dx^2}$ comes out <em>negative</em>, then

the value of $y$ which you got must be a <em>maximum</em>.
That's the rule.

<a name="figure34">
<p><img src="33283-t/images/126a.pdf.png-1.png">
<a name="figure35">
<p><img src="33283-t/images/126b.pdf.png-1.png">

<p>The reason of it ought to be quite evident. Think
of any curve that has a minimum point in it (like
<a href="10.html#figure15">Figure 15</a>), or like <a href="#figure34">Figure 34</a>, where the point of
minimum $y$ is marked $M$, and the curve is <em>concave</em>
upwards. To the left of $M$ the slope is downward,
that is, negative, and is getting less negative. To the
right of $M$ the slope has become upward, and is
getting more and more upward. Clearly the change
of slope as the curve passes through $M$ is such that
$\dfrac{d^2y}{dx^2}$ is <em>positive</em>, for its operation, as $x$ increases toward
the right, is to convert a downward slope into an
upward one.

<p>Similarly, consider any curve that has a maximum
point in it (like <a href="10.html#figure16">Figure 16</a>), or like <a href="#figure35">Figure 35</a>, where
the curve is <em>convex</em>, and the maximum point is
marked $M$. In this case, as the curve passes through $M$
from left to right, its upward slope is converted

into a downward or negative slope, so that in this
case the �slope of the slope� $\dfrac{d^2y}{dx^2}$ is <em>negative</em>.

<p>Go back now to the examples of the last chapter
and verify in this way the conclusions arrived at as to
whether in any particular case there is a maximum
or a minimum. You will find below a few worked
out examples.